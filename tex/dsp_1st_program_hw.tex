\documentclass[UTF8,a4paper]{ctexart}
% \pagestyle{empty}
% \documentclass[a4paper]{article}
\usepackage{amssymb}
\usepackage{amsmath}
\usepackage{geometry}
\usepackage{graphicx}
\usepackage{float}
\usepackage{subfigure}

\usepackage{CJK}
\usepackage{layout}
\usepackage{lastpage}
\usepackage{fancyhdr}
\pagestyle{fancy} 
\lhead{无42 林子恒 2014011054}
\rhead{page \thepage\ of \pageref{LastPage}}
\cfoot{}


\usepackage{times}
\usepackage{fancybox}
\usepackage{xcolor}
\usepackage{listings}
\lstset{breaklines}
\lstset{language=C++}
\lstset{                        %Settings for listings package.
	language=[ANSI]{C},
         numbers=left,
         numberstyle=\tiny,
         basicstyle=\small\ttfamily,
         stringstyle=\color{purple},
         keywordstyle=\color{blue}\bfseries,
         commentstyle=\color{olive},
         directivestyle=\color{blue},
         frame=shadowbox,
         showspaces=false,
       %framerule=0pt,
       %backgroundcolor=\color{pink},
         rulesepcolor=\color{red!20!green!20!blue!20},
         tabsize=4,
      %rulesepcolor=\color{brown}
         xleftmargin=5em, xrightmargin=5em, aboveskip=1em
}


\geometry{left=2.5cm, right=2.5cm, top=2cm, bottom=2cm}


  \author{W42 林子恒 2014011054} 
  \title{DSP Programming Homework -- Fast Fourier Transform}
  \date{2014.10.24}

\CTEXsetup[format={\Large\bfseries}]{section}


%%%%%%%%%%%%%%%%%%%%%%%%%%%%%%%%%%%%
\begin{document}
  \maketitle
  \thispagestyle{empty}

\tableofcontents

\newpage


%%%%%%%%%%%%%%%%%%%%%%%%%%%%%%%%%%%%
\section{Problem}

编写任意2的证书次幂点数的基2 DIT-FFT 和 基2 DIF-FFT 通用 c/c++ 程序,并与直接计算DFT比较点数 $2^{N}$ (N=10,...,16) 时运行时间的差异。


%%%%%%%%%%%%%%%%%%%%%%%%%%%%%%%%%%%%
\section{Solution}

\subsection{Realization of Discrete Fourier Transform \& Fast Fourier Transform}

\subsubsection{Realization of Discrete Fourier Transform (dft.h)}

Nothing special, calculated following the definition of DFT.

Note that for this DFT method and the other FFT methods in this article later, we would first calculated all of the $W^{k}_{N}$ and store them in an array to save useless repeat computing time.

    \begin{lstlisting}
// DFT
	// input_seq[]: 
	// N: size of input_seq
complex* DFT(complex input_seq[], int N) {

	// Calc WN
	complex* WN = new complex[N];
	WN = Calc_WN(N);

	complex* DFTed_seq = new complex[N];
	for (int i = 0; i < N; ++i)
	{
		for (int j = 0; j < N; ++j)
		{
			int k_mod = (i*j) % N;
			DFTed_seq[i] = ComplexAdd( DFTed_seq[i], ComplexMul(input_seq[j], WN[k_mod]) );
		}
	}

	return DFTed_seq;
}
    \end{lstlisting}




\newpage
\subsubsection{Realization of Decimate-in-Time Fast Fourier Transform (dit\_fft.h)}


	For first step, the function DIT\_FFT\_reordered() would call reorder\_seq() to reorder the input sequence to bit-reversed order which is required by the DIT-FFT method.
	
	For second step, we call the function DIT\_FFT() to recursively calculate the DFT of the sequence.

    \begin{lstlisting}
complex* DIT_FFT_reordered(complex input_seq[], int N);
complex* DIT_FFT(complex input_seq[], int N, complex WN[], int recur_time_count);

// DIT-FFT
	// input_seq[]: 
	// N: size of input_seq
		// Must be a 2^k integer
complex* DIT_FFT_reordered(complex input_seq[], int N) {
	// Initialize
	complex* reordered_seq = new complex[N];
	// Calc WN
	complex* WN = new complex[N];
	WN = Calc_WN(N);
	// Reorder
	reordered_seq = reorder_seq(input_seq, N);
	// Calc DIF-FFT
	reordered_seq = DIT_FFT(reordered_seq, N, WN, 0);
	return reordered_seq;
}
complex* DIT_FFT(complex input_seq[], int N, complex WN[], int recur_time_count) {
	// cout << "\tDIF_FFT executed!\n"; // for validation
	// output seq
	complex* return_seq = new complex[N];
	if ( N != 2 ) {
		int k = pow(2,recur_time_count);
		// Split input_seq into 2
		complex* first_half_input_seq = new complex[N/2];
		complex* second_half_input_seq = new complex[N/2];
		for (int i = 0; i < N/2; ++i) {
			first_half_input_seq[i] = input_seq[i];
		}
		for (int i = 0; i < N/2; ++i) {
			second_half_input_seq[i] = input_seq[i+N/2];
		}
		// DFT
		complex* DFTed_first_half_seq = new complex[N/2];
		DFTed_first_half_seq = DIT_FFT(first_half_input_seq, N/2, WN, recur_time_count+1);
		complex* DFTed_second_half_seq = new complex[N/2];
		DFTed_second_half_seq = DIT_FFT(second_half_input_seq, N/2, WN, recur_time_count+1);
		// Calc
		complex* output_first_half_seq = new complex[N/2];
		complex* output_second_half_seq = new complex[N/2];
		for (int i = 0; i < N/2; ++i) {
			output_first_half_seq[i] = ComplexAdd(DFTed_first_half_seq[i], ComplexMul(DFTed_second_half_seq[i], WN[i*k]) ) ;
		}
		for (int i = 0; i < N/2; ++i) {
			output_second_half_seq[i] = ComplexAdd(DFTed_first_half_seq[i], ComplexMul( ReverseComplex(DFTed_second_half_seq[i]), WN[i*k] ) );
		}
		// Append [output_first_half_seq] & [output_second_half_seq]
		return_seq = append_seq(output_first_half_seq, output_second_half_seq, N/2);
		return return_seq;
	} else if ( N == 2 ) { // Smallest Butterfly Unit
		// cout << "\tDIT_FFT N==2 triggered!\n"; // for validation
		return_seq[0] = ComplexAdd(input_seq[0], ComplexMul(input_seq[1], WN[0]) );
		return_seq[1] = ComplexAdd(input_seq[0], ComplexMul( ReverseComplex(input_seq[1]), WN[0] ) );
		return return_seq;
	}
	// return [return_seq] # unordered
	return return_seq;
}

    \end{lstlisting}



\newpage
\subsubsection{Realization of Decimate-in-Frequency Fast Fourier Transform (dif\_fft.h)}

	For first step, the function DIT\_FFT\_reordered() would call the function DIT\_FFT() to recursively calculate the DFT of the sequence.
	
	For second step, we call reorder\_seq() to reorder the input sequence to bit-reversed order which is required by the DIF-FFT method.


    \begin{lstlisting}
complex* DIF_FFT_reordered(complex input_seq[], int N);
complex* DIF_FFT(complex input_seq[], int N, complex WN[], int recur_time_count);
// DIF-FFT
	// input_seq[]: 
	// N: size of input_seq
		// Must be a 2^k integer
complex* DIF_FFT_reordered(complex input_seq[], int N) {
	// Initialize
	complex* reordered_seq = new complex[N];
	// Calc WN
	complex* WN = new complex[N];
	WN = Calc_WN(N);
	// Calc DIF-FFT
	reordered_seq = DIF_FFT(input_seq, N, WN, 0);
	// Reorder
	reordered_seq = reorder_seq(reordered_seq, N);
	return reordered_seq;
}
complex* DIF_FFT(complex input_seq[], int N, complex WN[], int recur_time_count) {
	// cout << "\tDIF_FFT executed!\n"; // for validation
	// output seq
	complex* return_seq = new complex[N];
	if ( N != 2 ) {
		complex* first_half_seq = new complex[N/2];
		complex* second_half_seq = new complex[N/2];
		int k = pow(2,recur_time_count);
		// Calc
		for (int i = 0; i < N/2; ++i) {
			first_half_seq[i] = ComplexAdd(input_seq[i], input_seq[i+N/2]) ;
		}
		for (int i = 0; i < N/2; ++i) {
			second_half_seq[i] = ComplexMul( ComplexAdd(input_seq[i], ReverseComplex(input_seq[i+N/2])), WN[i*k] ) ;
		}
		// DFT
		complex* DFTed_first_half_seq = new complex[N/2];
		DFTed_first_half_seq = DIF_FFT(first_half_seq, N/2, WN, recur_time_count+1);
		complex* DFTed_second_half_seq = new complex[N/2];
		DFTed_second_half_seq = DIF_FFT(second_half_seq, N/2, WN, recur_time_count+1);
		// Append [DFTed_first_half_seq] & [DFTed_second_half_seq]
		return_seq = append_seq(DFTed_first_half_seq, DFTed_second_half_seq, N/2);
		return return_seq;
	} else if ( N == 2 ) { // Smallest Butterfly Unit
		// cout << "\tDIF_FFT N==2 triggered!\n"; // for validation
		return_seq[0] = ComplexAdd(input_seq[0], input_seq[1]);
		return_seq[1] = ComplexMul( ComplexAdd(input_seq[0], ReverseComplex(input_seq[1])), WN[0] );
		return return_seq;
	}
	// return [return_seq] # unordered
	return return_seq;
}

    \end{lstlisting}



\newpage
\subsection{Some Other Supporting Functions}

	\subsubsection{Struct of Complex (complex.h)}
    \begin{lstlisting}
// Complex Struct
typedef struct Complex
{
	double re;
	double im;
	Complex() {
		re = 0;
		im = 0;
	};
	Complex(double a,double b) {
		re = a;
		im = b;
	};
} complex;
complex* append_seq(complex seq_1[], complex seq_2[], int N);
complex* reorder_seq(complex input_seq[], int N);
complex* Calc_WN(int N);
int reverse_bit(int value, int N);
// Multiplier
complex ComplexMul(complex c1, complex c2)
{
	complex r;
	r.re = c1.re*c2.re - c1.im*c2.im;
	r.im = c1.re*c2.im + c1.im*c2.re;
	return r;
}
// Adder
complex ComplexAdd(complex c1, complex c2)
{
	complex r;
	r.re = c1.re + c2.re;
	r.im = c1.im + c2.im;
	return r;
}
// -c
complex ReverseComplex(complex c)
{
	c.re = -c.re;
	c.im = -c.im;
	return c;
}

    \end{lstlisting}

\newpage
	\subsubsection{Calculation of $W^{k}_{N}$ (complex.h - Calc\_WN())}
    \begin{lstlisting}
// Calc WN[], with N = input_N
complex* Calc_WN(int N) {

	cout << "Calculating WN[] of N = " << N << " ... \t";
	complex* WN = new complex[N];

	complex WN_unit; WN_unit.re = cos(2*PI/N); WN_unit.im = -sin(2*PI/N);
	WN[0].re=1; WN[0].im=0;

	for (int i = 1; i < N; ++i)
	{
		WN[i] = ComplexMul(WN[i-1], WN_unit);
	}

	return WN;
}

    \end{lstlisting}

	\subsubsection{Convert to Bit-reversed order (complex.h - reverse\_bit())}
    \begin{lstlisting}
// Reverse Bit
	// input: 
		// a decimal num, 
		// N-based reverse method
	// output: a decimal num
int reverse_bit(int value, int N) {

	int ret = 0;
	int i = 0;

	while (i < N) {
		ret <<= 1;
		ret |= (value>>i) & 1;
		i++;
	}

	return ret;
}

    \end{lstlisting}

\newpage
	\subsubsection{Validate \& Evaluate Function (validate\_n\_evaluate.h)}

	The Code is too long and less important, please see validate\_n\_evaluate.h

	I realize two functions in this file to be called by the main() function.

    \begin{lstlisting}
void validate_result(complex input_seq[], int N, int dft_dit_dif);
void evaluate_performance(complex input_seq[], int N_max, int dft_dit_dif);
    \end{lstlisting}


	\subsubsection{Main Function (fft.cpp)}
    \begin{lstlisting}
# include "complex.h"	// definition of struct complex, Calc of WN[]
# include "dft.h"		// DFT
# include "dit_fft.h"	// DIT-FFT
# include "dif_fft.h"	// DIF-FFT
# include "validate_n_evaluate.h"

int main(int argc, char ** argv) 
{
	// Get argv
		int N_max = atoi(argv[1]); // length of input sequence
			// input 7 to run 2^{10,11,12,13,14,15,16}
			// input 6 to run 2^{10,11,12,13,14,15}
		int validate_or_evaluate = atoi(argv[2]); 
			// 1 for validate, 0 for ignore
		int dft_dit_dif = atoi(argv[3]); 
			// 1:DFT, 2:DIT, 3:DIF, 4:To compute everything for validation
	// Initialize
	// Setup input sequence
		complex* input_seq = new complex[N_max];
		input_seq[0] = complex(1,0);
	// For validation of the result
		if (validate_or_evaluate == 1) {
			validate_result(input_seq, N_max, dft_dit_dif);
			return 0;
		}
	// For compare the performance(run time) of DFT/DIT/DIF
		else if (validate_or_evaluate == 0) {
			evaluate_performance(input_seq, N_max, dft_dit_dif);
			return 0;
		}
	// end
	    return 0;
}



    \end{lstlisting}

   
\newpage
\subsection{Validation of Correctness}


Run the following script to obtain the result

    \begin{lstlisting}[language=sh]
make fft
./fft 8 1 4
    \end{lstlisting}


\begin{enumerate}
\item the 1st argument 8 indicate that we would calculate an 8-point DFT
\item the 2nd argument = 1 indicate that we are using validation mode
\item the 3rd argument = 4 indicate that we are using all 3 methods
\end{enumerate}   

And it follows with the result:

    \begin{lstlisting}[language=sh]
Lin, Tzu-Heng's Work, 2014011054, W42
Dept. of Electronic Enigeering, Tsinghua University
Starting, This project is to calc DFT in Original-DFT / DIT-FFT / DIF-FFT...
	For Usage, Please see 'fft.cpp' 

Calculating DFT...
	X[0] = 2 + j*0
	X[1] = 1.70711 + j*-0.707107
	X[2] = 1 + j*-1
	X[3] = 0.292893 + j*-0.707107
	X[4] = 1.33227e-15 + j*-5.35898e-08
	X[5] = 0.292893 + j*0.707107
	X[6] = 1 + j*1
	X[7] = 1.70711 + j*0.707107
Calculating DIT-FFT...
	X[0] = 2 + j*0
	X[1] = 1.70711 + j*-0.707107
	X[2] = 1 + j*-1
	X[3] = 0.292893 + j*-0.707107
	X[4] = 0 + j*0
	X[5] = 0.292893 + j*0.707107
	X[6] = 1 + j*1
	X[7] = 1.70711 + j*0.707107
Calculating DIF-FFT...
	X[0] = 2 + j*0
	X[1] = 1.70711 + j*-0.707107
	X[2] = 1 + j*-1
	X[3] = 0.292893 + j*-0.707107
	X[4] = 0 + j*0
	X[5] = 0.292893 + j*0.707107
	X[6] = 1 + j*1
	X[7] = 1.70711 + j*0.707107
    \end{lstlisting}

We are using a test sample x[n] = \{1,1,0,0,0,0,0,0\}, we can validate that the result is correct.




\newpage
\subsection{Performance Comparision of Algorithms}


Run the following script to obtain the result

    \begin{lstlisting}[language=sh]
make fft
./fft 7 0 4	# Calc 2^{10~16} using all 3 methods
# ./fft 6 0 4	# Calc 2^{10~15} using all 3 methods
    \end{lstlisting}
    
Note that:

\begin{enumerate}
\item the 1st argument 7 indicate that we would calculate all 7 N's from $2^{10}$ to $2^{16}$:
\item the 2nd argument = 0 indicate that we are using performance evaluation mode
\item the 3rd argument = 4 indicate that we are using all 3 methods
\end{enumerate}    

And it follows with the result:

\begin{lstlisting}[language=sh]
Lin, Tzu-Heng's Work, 2014011054, W42
Dept. of Electronic Enigeering, Tsinghua University

This project is to calc DFT in Original-DFT / DIT-FFT / DIF-FFT...
	For Usage, Please see 'fft.cpp' 

Calculating DIT-FFT...
N = 1024	Run time = 2 ms
N = 2048	Run time = 6 ms
N = 4096	Run time = 14 ms
N = 8192	Run time = 24 ms
N = 16384	Run time = 53 ms
N = 32768	Run time = 106 ms
N = 65536	Run time = 200 ms
Calculating DIF-FFT...
N = 1024	Run time = 1 ms
N = 2048	Run time = 3 ms
N = 4096	Run time = 7 ms
N = 8192	Run time = 16 ms
N = 16384	Run time = 39 ms
N = 32768	Run time = 79 ms
N = 65536	Run time = 154 ms
Calculating DFT...
N = 1024	Run time = 38 ms
N = 2048	Run time = 180 ms
N = 4096	Run time = 771 ms
N = 8192	Run time = 2948 ms
N = 16384	Run time = 12811 ms
N = 32768	Run time = 61836 ms
N = 65536	Run time = 356046 ms
\end{lstlisting}
    
We can see that when N = $2^{16}$, the DFT algorithm can be more than 2000 time slower than the FFT algorithm. And this number grows when N continues to become bigger.

  
  
\end{document}